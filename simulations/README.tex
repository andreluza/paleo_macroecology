% Options for packages loaded elsewhere
\PassOptionsToPackage{unicode}{hyperref}
\PassOptionsToPackage{hyphens}{url}
%
\documentclass[
]{article}
\usepackage{amsmath,amssymb}
\usepackage{iftex}
\ifPDFTeX
  \usepackage[T1]{fontenc}
  \usepackage[utf8]{inputenc}
  \usepackage{textcomp} % provide euro and other symbols
\else % if luatex or xetex
  \usepackage{unicode-math} % this also loads fontspec
  \defaultfontfeatures{Scale=MatchLowercase}
  \defaultfontfeatures[\rmfamily]{Ligatures=TeX,Scale=1}
\fi
\usepackage{lmodern}
\ifPDFTeX\else
  % xetex/luatex font selection
\fi
% Use upquote if available, for straight quotes in verbatim environments
\IfFileExists{upquote.sty}{\usepackage{upquote}}{}
\IfFileExists{microtype.sty}{% use microtype if available
  \usepackage[]{microtype}
  \UseMicrotypeSet[protrusion]{basicmath} % disable protrusion for tt fonts
}{}
\makeatletter
\@ifundefined{KOMAClassName}{% if non-KOMA class
  \IfFileExists{parskip.sty}{%
    \usepackage{parskip}
  }{% else
    \setlength{\parindent}{0pt}
    \setlength{\parskip}{6pt plus 2pt minus 1pt}}
}{% if KOMA class
  \KOMAoptions{parskip=half}}
\makeatother
\usepackage{xcolor}
\usepackage[margin=1in]{geometry}
\usepackage{color}
\usepackage{fancyvrb}
\newcommand{\VerbBar}{|}
\newcommand{\VERB}{\Verb[commandchars=\\\{\}]}
\DefineVerbatimEnvironment{Highlighting}{Verbatim}{commandchars=\\\{\}}
% Add ',fontsize=\small' for more characters per line
\usepackage{framed}
\definecolor{shadecolor}{RGB}{248,248,248}
\newenvironment{Shaded}{\begin{snugshade}}{\end{snugshade}}
\newcommand{\AlertTok}[1]{\textcolor[rgb]{0.94,0.16,0.16}{#1}}
\newcommand{\AnnotationTok}[1]{\textcolor[rgb]{0.56,0.35,0.01}{\textbf{\textit{#1}}}}
\newcommand{\AttributeTok}[1]{\textcolor[rgb]{0.13,0.29,0.53}{#1}}
\newcommand{\BaseNTok}[1]{\textcolor[rgb]{0.00,0.00,0.81}{#1}}
\newcommand{\BuiltInTok}[1]{#1}
\newcommand{\CharTok}[1]{\textcolor[rgb]{0.31,0.60,0.02}{#1}}
\newcommand{\CommentTok}[1]{\textcolor[rgb]{0.56,0.35,0.01}{\textit{#1}}}
\newcommand{\CommentVarTok}[1]{\textcolor[rgb]{0.56,0.35,0.01}{\textbf{\textit{#1}}}}
\newcommand{\ConstantTok}[1]{\textcolor[rgb]{0.56,0.35,0.01}{#1}}
\newcommand{\ControlFlowTok}[1]{\textcolor[rgb]{0.13,0.29,0.53}{\textbf{#1}}}
\newcommand{\DataTypeTok}[1]{\textcolor[rgb]{0.13,0.29,0.53}{#1}}
\newcommand{\DecValTok}[1]{\textcolor[rgb]{0.00,0.00,0.81}{#1}}
\newcommand{\DocumentationTok}[1]{\textcolor[rgb]{0.56,0.35,0.01}{\textbf{\textit{#1}}}}
\newcommand{\ErrorTok}[1]{\textcolor[rgb]{0.64,0.00,0.00}{\textbf{#1}}}
\newcommand{\ExtensionTok}[1]{#1}
\newcommand{\FloatTok}[1]{\textcolor[rgb]{0.00,0.00,0.81}{#1}}
\newcommand{\FunctionTok}[1]{\textcolor[rgb]{0.13,0.29,0.53}{\textbf{#1}}}
\newcommand{\ImportTok}[1]{#1}
\newcommand{\InformationTok}[1]{\textcolor[rgb]{0.56,0.35,0.01}{\textbf{\textit{#1}}}}
\newcommand{\KeywordTok}[1]{\textcolor[rgb]{0.13,0.29,0.53}{\textbf{#1}}}
\newcommand{\NormalTok}[1]{#1}
\newcommand{\OperatorTok}[1]{\textcolor[rgb]{0.81,0.36,0.00}{\textbf{#1}}}
\newcommand{\OtherTok}[1]{\textcolor[rgb]{0.56,0.35,0.01}{#1}}
\newcommand{\PreprocessorTok}[1]{\textcolor[rgb]{0.56,0.35,0.01}{\textit{#1}}}
\newcommand{\RegionMarkerTok}[1]{#1}
\newcommand{\SpecialCharTok}[1]{\textcolor[rgb]{0.81,0.36,0.00}{\textbf{#1}}}
\newcommand{\SpecialStringTok}[1]{\textcolor[rgb]{0.31,0.60,0.02}{#1}}
\newcommand{\StringTok}[1]{\textcolor[rgb]{0.31,0.60,0.02}{#1}}
\newcommand{\VariableTok}[1]{\textcolor[rgb]{0.00,0.00,0.00}{#1}}
\newcommand{\VerbatimStringTok}[1]{\textcolor[rgb]{0.31,0.60,0.02}{#1}}
\newcommand{\WarningTok}[1]{\textcolor[rgb]{0.56,0.35,0.01}{\textbf{\textit{#1}}}}
\usepackage{graphicx}
\makeatletter
\def\maxwidth{\ifdim\Gin@nat@width>\linewidth\linewidth\else\Gin@nat@width\fi}
\def\maxheight{\ifdim\Gin@nat@height>\textheight\textheight\else\Gin@nat@height\fi}
\makeatother
% Scale images if necessary, so that they will not overflow the page
% margins by default, and it is still possible to overwrite the defaults
% using explicit options in \includegraphics[width, height, ...]{}
\setkeys{Gin}{width=\maxwidth,height=\maxheight,keepaspectratio}
% Set default figure placement to htbp
\makeatletter
\def\fps@figure{htbp}
\makeatother
\setlength{\emergencystretch}{3em} % prevent overfull lines
\providecommand{\tightlist}{%
  \setlength{\itemsep}{0pt}\setlength{\parskip}{0pt}}
\setcounter{secnumdepth}{-\maxdimen} % remove section numbering
\ifLuaTeX
  \usepackage{selnolig}  % disable illegal ligatures
\fi
\IfFileExists{bookmark.sty}{\usepackage{bookmark}}{\usepackage{hyperref}}
\IfFileExists{xurl.sty}{\usepackage{xurl}}{} % add URL line breaks if available
\urlstyle{same}
\hypersetup{
  pdftitle={Spatiotemporal dynamics of Cynodontia through end-Paleozoic and Mesozoic eras},
  pdfauthor={ALLuza, MG Bender, CS Dambros, Pretto F, L Kerber - Departamento de Ecologia e Evolução, Universidade Federal de Santa Maria},
  hidelinks,
  pdfcreator={LaTeX via pandoc}}

\title{Spatiotemporal dynamics of Cynodontia through end-Paleozoic and
Mesozoic eras}
\author{ALLuza, MG Bender, CS Dambros, Pretto F, L Kerber - Departamento
de Ecologia e Evolução, Universidade Federal de Santa Maria}
\date{2023-07-03}

\begin{document}
\maketitle

\hypertarget{supporting-information-s1}{%
\subsection{Supporting Information S1}\label{supporting-information-s1}}

We made a simulation experiment to test whether the model can
effectively estimate origination and extinction probabilities across
sites. The simulations were based on simulations of the state-space
dynamic model described in Royle \& Kery 2021 (AHM Book V2).

\hypertarget{single-taxa-simulations}{%
\subsection{Single-taxa simulations}\label{single-taxa-simulations}}

We simulated the dynamics in the occurrence of one simulated taxon at
twelve sites and 30 intervals of time (`nyears'). The probability of
origination (`gamma') in one site over subsequent intervals of time
represents both the probability of speciation (by phylogeny branching
processes) and colonization (through immigration from other sites). The
range of `gamma' values depict how much this parameter will change over
time in the simulations. Origination started at either high or low
values depending on whether the site was optimal or not, and tended to
drop to zero at the last time.

\begin{Shaded}
\begin{Highlighting}[]
\CommentTok{\# set seed for simulation}
\FunctionTok{set.seed}\NormalTok{(}\DecValTok{123}\NormalTok{)}

\NormalTok{nyears }\OtherTok{\textless{}{-}} \DecValTok{30} \CommentTok{\# number of time bins (stages)}

\CommentTok{\# values of gamma and phi}
\CommentTok{\# range of values for gamma per site}
\CommentTok{\# the range of values depict how much gamma will change over time}
\NormalTok{gamma }\OtherTok{\textless{}{-}} \FunctionTok{list}\NormalTok{(}\FunctionTok{c}\NormalTok{(}\FloatTok{0.05}\NormalTok{,}\DecValTok{0}\NormalTok{), }\CommentTok{\# suboptimal site}
              \FunctionTok{c}\NormalTok{(}\FloatTok{0.05}\NormalTok{,}\DecValTok{0}\NormalTok{), }\CommentTok{\# suboptimal site}
              \FunctionTok{c}\NormalTok{(}\FloatTok{0.10}\NormalTok{,}\DecValTok{0}\NormalTok{), }\CommentTok{\# suboptimal site}
              \FunctionTok{c}\NormalTok{(}\FloatTok{0.150}\NormalTok{,}\DecValTok{0}\NormalTok{), }\CommentTok{\# suboptimal site}
              \FunctionTok{c}\NormalTok{(}\FloatTok{0.45}\NormalTok{,}\DecValTok{0}\NormalTok{), }\CommentTok{\# intermediate site}
              \FunctionTok{c}\NormalTok{(}\FloatTok{0.49}\NormalTok{,}\DecValTok{0}\NormalTok{), }\CommentTok{\# intermediate site}
              \FunctionTok{c}\NormalTok{(}\FloatTok{0.5}\NormalTok{,}\DecValTok{0}\NormalTok{), }\CommentTok{\# intermediate site}
              \FunctionTok{c}\NormalTok{(}\FloatTok{0.55}\NormalTok{,}\DecValTok{0}\NormalTok{), }\CommentTok{\# intermediate site}
              \FunctionTok{c}\NormalTok{(}\FloatTok{0.65}\NormalTok{,}\DecValTok{0}\NormalTok{), }\CommentTok{\# optimal site }
              \FunctionTok{c}\NormalTok{(}\FloatTok{0.75}\NormalTok{,}\DecValTok{0}\NormalTok{), }\CommentTok{\# optimal site}
              \FunctionTok{c}\NormalTok{(}\FloatTok{0.85}\NormalTok{,}\DecValTok{0}\NormalTok{), }\CommentTok{\# optimal site}
              \FunctionTok{c}\NormalTok{(}\FloatTok{0.95}\NormalTok{,}\DecValTok{0}\NormalTok{)) }\CommentTok{\# optimal site}
\NormalTok{phi }\OtherTok{\textless{}{-}} \FunctionTok{c}\NormalTok{(}\FloatTok{0.1}\NormalTok{,}\FloatTok{0.05}\NormalTok{) }\CommentTok{\# small persistence probability, should decrease over time}
\NormalTok{psi1}\OtherTok{\textless{}{-}}\FloatTok{0.15} \CommentTok{\# initial occupancy}
\end{Highlighting}
\end{Shaded}

In this simulation, `phi' is the probability of persistence over time
(note that extinction probability is 1-phi), and `psi1' is the
proportion of occupied sites at t=1. The `phi' was very low and its
values should decrease over time, from 0.1 to 0.05. We choose these
values to represent the highly dynamic nature of paleontological
datasets.

Then we used these parameters to simulate a dataset per site. We applied
the function \emph{lapply} to the \emph{simDynocc} function of the
`AHMbook' package to simulate one dataset per site, for all values of
`gamma'. Note that the dataset was simulated under a condition of
perfect detection (range.p = c(1,1)) in a number of surveys equal to 1.

\begin{Shaded}
\begin{Highlighting}[]
\CommentTok{\# simulate dataset}
\FunctionTok{require}\NormalTok{(}\StringTok{"AHMbook"}\NormalTok{)}
\FunctionTok{require}\NormalTok{(dplyr)}
\FunctionTok{require}\NormalTok{(ggplot2)}

\NormalTok{sim\_data }\OtherTok{\textless{}{-}} \FunctionTok{lapply}\NormalTok{ (gamma, }\ControlFlowTok{function}\NormalTok{ (i) }
  
  
            \FunctionTok{simDynocc}\NormalTok{(}\AttributeTok{nsites=} \DecValTok{1}\NormalTok{, }\CommentTok{\# 1 site each time}
                      \AttributeTok{nyears=}\NormalTok{nyears,}
                      \AttributeTok{nsurveys=}\DecValTok{1}\NormalTok{,}
                      \AttributeTok{mean.psi1=}\NormalTok{psi1,}
                      \AttributeTok{range.phi =}\NormalTok{ phi,}
                      \AttributeTok{range.gamma =}\NormalTok{i,}
                      \AttributeTok{range.p=}\FunctionTok{c}\NormalTok{(}\DecValTok{1}\NormalTok{,}\DecValTok{1}\NormalTok{), }\CommentTok{\# perfect detection}
                      \AttributeTok{show.plots =}\NormalTok{ F)}
\NormalTok{            )}
\end{Highlighting}
\end{Shaded}

From the simulation of each site, we extracted the `z' matrix which
depicts the true/realized species occurrence in each site and time
interval (thus `z' has 12 rows and 30 columns). We extracted the `z'
matrix using the function \emph{sapply} applied to the simulated data
for each combination of `gamma'.

\begin{Shaded}
\begin{Highlighting}[]
\CommentTok{\# melt data}
\NormalTok{sim\_data }\OtherTok{\textless{}{-}} \FunctionTok{t}\NormalTok{(}\FunctionTok{sapply}\NormalTok{ (sim\_data, }\StringTok{"[["}\NormalTok{, }\StringTok{"z"}\NormalTok{))}
\end{Highlighting}
\end{Shaded}

The simulated data produced the following dynamic in site occupancy
probability over time:

\begin{Shaded}
\begin{Highlighting}[]
\CommentTok{\#number of sites occupied over time}
\FunctionTok{plot}\NormalTok{ (}\FunctionTok{colSums}\NormalTok{(sim\_data)}\SpecialCharTok{/}\FunctionTok{length}\NormalTok{(gamma), }\AttributeTok{type=}\StringTok{"b"}\NormalTok{, }\AttributeTok{ylab =} \StringTok{"Proportion of Occupied Sites"}\NormalTok{, }\AttributeTok{xlab =} \StringTok{"Time"}\NormalTok{)}
\end{Highlighting}
\end{Shaded}

\includegraphics{README_files/figure-latex/unnamed-chunk-3-1.pdf}

The simulations start with a small proportion of occupied sites at
interval t=1, varies substantially over time, and then tends to a few
number of occupied sites at t=30. Note that the taxon sometimes
disappeared (i.e., proportion of occupied sites = 0), and then can
originate again through speciation or colonization.

Then we design the dynamic model, bundle data, choose parameters we want
to track across the Monte Carlo Markov Chains (MCMC) samples, set MCMC
settings, and run the model using JAGS (Bayesian framework to estimate
model parameters).

\begin{Shaded}
\begin{Highlighting}[]
\FunctionTok{cat}\NormalTok{(}\FunctionTok{readLines}\NormalTok{(}\StringTok{\textquotesingle{}dyn\_model\_vectorized\_no\_detection.txt\textquotesingle{}}\NormalTok{), }\AttributeTok{sep =} \StringTok{\textquotesingle{}}\SpecialCharTok{\textbackslash{}n}\StringTok{\textquotesingle{}}\NormalTok{)}
\end{Highlighting}
\end{Shaded}

\begin{verbatim}

   
     model {
    
     #############################################################
    #                                                           #
    #                  Biological process                       #
    #                                                           #
    #############################################################
    
    
    # Site Occupancy Dynamics (Priors)
    for(i in 1:nsites) {
      gamma[i]~dunif(0,1)
      phi[i]~dunif(0,1)
    }
    
    psi1~dunif(0,1)     
         
    # Initial values
    for (i in 1:nsites) {
      p[i,1] <- 0.15    # Initial occupancy probability
      }
    

   ############      Model       #############
   for (i in 1:nsites) {
      
          y[i,1] ~ dbern(psi1) # Initial occupancy probability
    
              for (t in 2:nint){
            
                # model likelihood
                ### modeling dynamics conditional on previous time observed occurrence y
                
                muZ[i,t] <- y[i,t-1] *  phi[i] + ### if occupied, p of not getting extinct/persist in the next time
                          (1-y[i,t-1]) *  gamma[i] ###  if not occupied, p of originate in the next time
                
        # realized occurrence
            p[i,t] ~ dbern(muZ[i,t])      # Occupancy model
            y[i,t] ~ dbern(p[i,t])        # Observation model
    
        }#t
      
     }#i
    
    
    
    # derived params
    
    for (t in 1:nint){
    
      ISS[t]<-sum(y[,t])
    
   }
    
    }## end of the model
    
    
    
    
\end{verbatim}

\begin{Shaded}
\begin{Highlighting}[]
\DocumentationTok{\#\# bundle data}
\FunctionTok{str}\NormalTok{(jags.data }\OtherTok{\textless{}{-}} \FunctionTok{list}\NormalTok{(}\AttributeTok{y =}\NormalTok{ sim\_data,}
                      \AttributeTok{nsites =} \FunctionTok{dim}\NormalTok{(sim\_data)[}\DecValTok{1}\NormalTok{],}
                      \AttributeTok{nint=} \FunctionTok{dim}\NormalTok{(sim\_data)[}\DecValTok{2}\NormalTok{],}
                      \AttributeTok{psinit=}\NormalTok{psi1)}
\NormalTok{)}
\end{Highlighting}
\end{Shaded}

\begin{verbatim}
TRUE List of 4
TRUE  $ y     : int [1:12, 1:30] 0 0 0 0 0 0 1 0 0 0 ...
TRUE  $ nsites: int 12
TRUE  $ nint  : int 30
TRUE  $ psinit: num 0.15
\end{verbatim}

\begin{Shaded}
\begin{Highlighting}[]
\DocumentationTok{\#\# Parameters to monitor}
\DocumentationTok{\#\# long form}
\NormalTok{params }\OtherTok{\textless{}{-}} \FunctionTok{c}\NormalTok{(}
  
  \StringTok{"psi1"}\NormalTok{,}
  \StringTok{"gamma"}\NormalTok{, }
  \StringTok{"phi"}\NormalTok{,}
  \StringTok{"p"}\NormalTok{,}
  \StringTok{"y"}
  
\NormalTok{)}

\DocumentationTok{\#\# MCMC settings}
\DocumentationTok{\#\#\#\#\#\#\#\#\#\#\#\#\#\#\#\#\#\#\#\#\#\#}
\NormalTok{na }\OtherTok{\textless{}{-}} \DecValTok{1000}\NormalTok{; nb }\OtherTok{\textless{}{-}} \DecValTok{5000}\NormalTok{; ni }\OtherTok{\textless{}{-}} \DecValTok{10000}\NormalTok{; nc }\OtherTok{\textless{}{-}} \DecValTok{3}\NormalTok{; nt }\OtherTok{\textless{}{-}} \DecValTok{10}

\CommentTok{\# MCMC runs}
\CommentTok{\# models}
\FunctionTok{require}\NormalTok{(jagsUI)}
\end{Highlighting}
\end{Shaded}

\begin{verbatim}
TRUE Carregando pacotes exigidos: jagsUI
\end{verbatim}

\begin{Shaded}
\begin{Highlighting}[]
\NormalTok{samples\_sims\_no\_det }\OtherTok{\textless{}{-}} \FunctionTok{jags}\NormalTok{ (}\AttributeTok{data =}\NormalTok{ jags.data, }
                             \AttributeTok{parameters.to.save =}\NormalTok{ params, }
                             \AttributeTok{model.file =} \StringTok{"dyn\_model\_vectorized\_no\_detection.txt"}\NormalTok{, }
                             \AttributeTok{inits =} \ConstantTok{NULL}\NormalTok{, }
                             \AttributeTok{n.chains =}\NormalTok{ nc, }
                             \AttributeTok{n.thin =}\NormalTok{ nt, }
                             \AttributeTok{n.iter =}\NormalTok{ ni, }
                             \AttributeTok{n.burnin =}\NormalTok{ nb, }
                             \AttributeTok{DIC =}\NormalTok{ T,  }
                             \AttributeTok{parallel=}\NormalTok{F}
\NormalTok{)}
\end{Highlighting}
\end{Shaded}

\begin{verbatim}
TRUE 
TRUE Processing function input....... 
TRUE 
TRUE Done. 
TRUE 
\end{verbatim}

\begin{verbatim}
TRUE Warning in jags.model(file = model.file, data = data, inits = inits, n.chains =
TRUE n.chains, : Unused variable "psinit" in data
\end{verbatim}

\begin{verbatim}
TRUE Compiling model graph
TRUE    Resolving undeclared variables
TRUE    Allocating nodes
TRUE Graph information:
TRUE    Observed stochastic nodes: 360
TRUE    Unobserved stochastic nodes: 373
TRUE    Total graph size: 860
TRUE 
TRUE Initializing model
TRUE 
TRUE Adaptive phase..... 
TRUE Adaptive phase complete 
TRUE  
TRUE 
TRUE  Burn-in phase, 5000 iterations x 3 chains 
TRUE  
TRUE 
TRUE Sampling from joint posterior, 5000 iterations x 3 chains 
TRUE  
TRUE 
TRUE Calculating statistics.......
\end{verbatim}

\begin{verbatim}
TRUE Warning in doTryCatch(return(expr), name, parentenv, handler): At least one
TRUE Rhat value could not be calculated.
\end{verbatim}

\begin{verbatim}
TRUE 
TRUE Done.
\end{verbatim}

The results of this small experiment are shown below. In red we
presented the estimates from the model, and in black we show the range
of `gamma' per site. The different sites are presented in the Y axis,
and the estimated probabilities (with 95\% Credible Intervals (CI)) in
the Y axis.

Overall, estimates of origination probability were uncertain but their
95\% CI always included the range of values of `gamma' which produced
the dataset. A similar pattern was found for persistence probability
`phi', for which the range of values generating the dataset (black
retangle) was always within the 95\% CI of `phi'.

\begin{Shaded}
\begin{Highlighting}[]
\CommentTok{\# plot the data}
\NormalTok{df\_res }\OtherTok{\textless{}{-}} \FunctionTok{do.call}\NormalTok{(rbind,gamma)}
\CommentTok{\# bind estimates}
\NormalTok{df\_res }\OtherTok{\textless{}{-}} \FunctionTok{cbind}\NormalTok{ (df\_res,}
                 \AttributeTok{sites =} \FunctionTok{seq}\NormalTok{(}\DecValTok{1}\NormalTok{,}\FunctionTok{nrow}\NormalTok{(df\_res)) ,}
                 \AttributeTok{gamma =}\NormalTok{ samples\_sims\_no\_det}\SpecialCharTok{$}\NormalTok{mean}\SpecialCharTok{$}\NormalTok{gamma,}
                 \AttributeTok{uci =} \FunctionTok{apply}\NormalTok{(samples\_sims\_no\_det}\SpecialCharTok{$}\NormalTok{sims.list}\SpecialCharTok{$}\NormalTok{gamma,}\DecValTok{2}\NormalTok{,quantile,}\FloatTok{0.975}\NormalTok{),}
                \AttributeTok{lci =} \FunctionTok{apply}\NormalTok{(samples\_sims\_no\_det}\SpecialCharTok{$}\NormalTok{sims.list}\SpecialCharTok{$}\NormalTok{gamma,}\DecValTok{2}\NormalTok{,quantile,}\FloatTok{0.025}\NormalTok{))}
\FunctionTok{colnames}\NormalTok{(df\_res)[}\DecValTok{1}\SpecialCharTok{:}\DecValTok{2}\NormalTok{]}\OtherTok{\textless{}{-}}\FunctionTok{c}\NormalTok{(}\StringTok{"inf\_range"}\NormalTok{,}\StringTok{"sup\_range"}\NormalTok{)}

\CommentTok{\# plot res simulations}

\FunctionTok{ggplot}\NormalTok{ (}\AttributeTok{data=}\FunctionTok{data.frame}\NormalTok{ (df\_res),}
        \FunctionTok{aes}\NormalTok{ (}\AttributeTok{x=}\NormalTok{sites,}\AttributeTok{y=}\NormalTok{gamma)) }\SpecialCharTok{+} 
  
  \FunctionTok{geom\_point}\NormalTok{(}\AttributeTok{col=}\StringTok{"red"}\NormalTok{, }\AttributeTok{size=}\DecValTok{2}\NormalTok{)}\SpecialCharTok{+}
  
  \FunctionTok{geom\_errorbar}\NormalTok{(}\FunctionTok{aes}\NormalTok{ (}\AttributeTok{ymin=}\NormalTok{lci,}\AttributeTok{ymax=}\NormalTok{ uci),}\AttributeTok{col=}\StringTok{"red"}\NormalTok{,}\AttributeTok{width=}\FloatTok{0.2}\NormalTok{,}\AttributeTok{linewidth=}\FloatTok{1.3}\NormalTok{)}\SpecialCharTok{+}

  \FunctionTok{geom\_errorbar}\NormalTok{(}\FunctionTok{aes}\NormalTok{ (}\AttributeTok{ymax=}\NormalTok{inf\_range,}\AttributeTok{ymin=}\NormalTok{ sup\_range),}\AttributeTok{width=}\FloatTok{0.2}\NormalTok{)}\SpecialCharTok{+}
  
  \FunctionTok{theme\_bw}\NormalTok{() }\SpecialCharTok{+} 
  
  \FunctionTok{ylab}\NormalTok{ (}\StringTok{"Origination probability"}\NormalTok{) }\SpecialCharTok{+} 
  
  \FunctionTok{xlab}\NormalTok{ (}\StringTok{"Sites"}\NormalTok{) }\SpecialCharTok{+} 
  
  \FunctionTok{scale\_x\_continuous}\NormalTok{(}\AttributeTok{breaks =} \FunctionTok{seq}\NormalTok{(}\DecValTok{1}\NormalTok{, }\DecValTok{12}\NormalTok{, }\AttributeTok{by =} \DecValTok{1}\NormalTok{))}
\end{Highlighting}
\end{Shaded}

\includegraphics{README_files/figure-latex/unnamed-chunk-6-1.pdf}

\begin{Shaded}
\begin{Highlighting}[]
\CommentTok{\# plot res simulations}
\CommentTok{\# bind estimates}

\NormalTok{df\_res\_phi }\OtherTok{\textless{}{-}} \FunctionTok{cbind}\NormalTok{ (}\AttributeTok{sites =} \FunctionTok{seq}\NormalTok{(}\DecValTok{1}\NormalTok{,}\FunctionTok{nrow}\NormalTok{(df\_res)) ,}
                 \AttributeTok{phi =}\NormalTok{ samples\_sims\_no\_det}\SpecialCharTok{$}\NormalTok{mean}\SpecialCharTok{$}\NormalTok{phi,}
                 \AttributeTok{uci =} \FunctionTok{apply}\NormalTok{(samples\_sims\_no\_det}\SpecialCharTok{$}\NormalTok{sims.list}\SpecialCharTok{$}\NormalTok{phi,}\DecValTok{2}\NormalTok{,quantile,}\FloatTok{0.975}\NormalTok{),}
                 \AttributeTok{lci =} \FunctionTok{apply}\NormalTok{(samples\_sims\_no\_det}\SpecialCharTok{$}\NormalTok{sims.list}\SpecialCharTok{$}\NormalTok{phi,}\DecValTok{2}\NormalTok{,quantile,}\FloatTok{0.025}\NormalTok{))}

\CommentTok{\# plot}
\FunctionTok{ggplot}\NormalTok{ (}\AttributeTok{data=}\FunctionTok{data.frame}\NormalTok{ (df\_res\_phi),}
        \FunctionTok{aes}\NormalTok{ (}\AttributeTok{x=}\NormalTok{sites,}\AttributeTok{y=}\NormalTok{phi)) }\SpecialCharTok{+} 
  
  \FunctionTok{geom\_point}\NormalTok{(}\AttributeTok{col=}\StringTok{"red"}\NormalTok{, }\AttributeTok{size=}\DecValTok{3}\NormalTok{)}\SpecialCharTok{+}
  
  \FunctionTok{geom\_errorbar}\NormalTok{(}\FunctionTok{aes}\NormalTok{ (}\AttributeTok{ymin=}\NormalTok{lci,}\AttributeTok{ymax=}\NormalTok{ uci),}\AttributeTok{col=}\StringTok{"red"}\NormalTok{,}\AttributeTok{width=}\FloatTok{0.2}\NormalTok{,}\AttributeTok{linewidth=}\FloatTok{1.3}\NormalTok{)}\SpecialCharTok{+}
  
  \CommentTok{\#geom\_errorbar(aes (ymin= min(phi), ymax=max(phi)),width=0.2)+}
  \FunctionTok{geom\_rect}\NormalTok{(}\FunctionTok{aes}\NormalTok{(}\AttributeTok{xmin =} \DecValTok{1}\NormalTok{, }\AttributeTok{xmax =} \DecValTok{12}\NormalTok{, }\AttributeTok{ymin =} \FloatTok{0.05}\NormalTok{, }\AttributeTok{ymax =} \FloatTok{0.1}\NormalTok{),}\AttributeTok{alpha=}\FloatTok{0.1}\NormalTok{) }\SpecialCharTok{+}

  \FunctionTok{theme\_bw}\NormalTok{() }\SpecialCharTok{+} 
  
  \FunctionTok{ylab}\NormalTok{ (}\StringTok{"Persistence probability"}\NormalTok{) }\SpecialCharTok{+} 
  
  \FunctionTok{xlab}\NormalTok{ (}\StringTok{"Sites"}\NormalTok{) }\SpecialCharTok{+} 
  
  \FunctionTok{scale\_x\_continuous}\NormalTok{(}\AttributeTok{breaks =} \FunctionTok{seq}\NormalTok{(}\DecValTok{1}\NormalTok{, }\DecValTok{12}\NormalTok{, }\AttributeTok{by =} \DecValTok{1}\NormalTok{))}
\end{Highlighting}
\end{Shaded}

\includegraphics{README_files/figure-latex/unnamed-chunk-7-1.pdf}

\hypertarget{multi-taxa-simulations}{%
\subsection{Multi-taxa simulations}\label{multi-taxa-simulations}}

Now, we repeated the simulations using 100 taxa to check whether the
estimates of origination and persistence probability become less
uncertain. We will use the same values of `nyears', `gamma', `phi', and
`psi1' used in the previous analysis for a single taxon.

\begin{Shaded}
\begin{Highlighting}[]
\NormalTok{nspp }\OtherTok{\textless{}{-}} \DecValTok{100} \CommentTok{\# number of taxa}

\CommentTok{\# simulate data}
\NormalTok{sim\_data\_spp }\OtherTok{\textless{}{-}} \FunctionTok{lapply}\NormalTok{ (}\FunctionTok{seq}\NormalTok{ (}\DecValTok{1}\NormalTok{,nspp), }\ControlFlowTok{function}\NormalTok{ (k)}
  
      \FunctionTok{lapply}\NormalTok{ (gamma, }\ControlFlowTok{function}\NormalTok{ (i) }
  
  
            \FunctionTok{simDynocc}\NormalTok{(}\AttributeTok{nsites=} \DecValTok{1}\NormalTok{, }\CommentTok{\# 1 site each time}
                      \AttributeTok{nyears=}\NormalTok{nyears,}
                      \AttributeTok{nsurveys=}\DecValTok{1}\NormalTok{,}
                      \AttributeTok{mean.psi1=}\NormalTok{psi1,}
                      \AttributeTok{range.phi =}\NormalTok{ phi,}
                      \AttributeTok{range.gamma =}\NormalTok{i,}
                      \AttributeTok{range.p=}\FunctionTok{c}\NormalTok{(}\DecValTok{1}\NormalTok{,}\DecValTok{1}\NormalTok{), }\CommentTok{\# perfect detection}
                      \AttributeTok{show.plots =}\NormalTok{ F)}
\NormalTok{            )}

\NormalTok{)}

\CommentTok{\# melt data}
\NormalTok{sim\_data\_spp\_z }\OtherTok{\textless{}{-}} \FunctionTok{lapply}\NormalTok{ (sim\_data\_spp, }\ControlFlowTok{function}\NormalTok{ (i)}

  \FunctionTok{t}\NormalTok{(}\FunctionTok{sapply}\NormalTok{ (i, }\StringTok{"[["}\NormalTok{, }\StringTok{"z"}\NormalTok{))}
  
\NormalTok{  )}
\end{Highlighting}
\end{Shaded}

Plot of the proportion of occupied sites per species.

\begin{Shaded}
\begin{Highlighting}[]
\CommentTok{\#number of sites occupied over time}
\FunctionTok{plot}\NormalTok{ (}\FunctionTok{colSums}\NormalTok{(sim\_data\_spp\_z[[}\DecValTok{1}\NormalTok{]])}\SpecialCharTok{/}\FunctionTok{length}\NormalTok{(gamma), }\AttributeTok{type=}\StringTok{"b"}\NormalTok{, }
      \AttributeTok{ylab =} \StringTok{"Proportion of Occupied Sites"}\NormalTok{, }\AttributeTok{xlab =} \StringTok{"Time"}\NormalTok{,}
      \AttributeTok{ylim=}\FunctionTok{c}\NormalTok{(}\DecValTok{0}\NormalTok{,}\DecValTok{1}\NormalTok{))}
\CommentTok{\# plot trend for each spp}
\FunctionTok{lapply}\NormalTok{ (sim\_data\_spp\_z, }\ControlFlowTok{function}\NormalTok{ (i)}

  \FunctionTok{lines}\NormalTok{ (}\FunctionTok{seq}\NormalTok{(}\DecValTok{1}\NormalTok{,nyears),}
        \FunctionTok{colSums}\NormalTok{(i)}\SpecialCharTok{/}\FunctionTok{length}\NormalTok{(gamma),}
  \AttributeTok{col =} \FunctionTok{rgb}\NormalTok{ (}\DecValTok{1}\NormalTok{,}\DecValTok{1}\NormalTok{,}\DecValTok{1}\NormalTok{,}\AttributeTok{alpha=}\FloatTok{0.1}\NormalTok{))}
  
\NormalTok{)}
\end{Highlighting}
\end{Shaded}

\includegraphics{README_files/figure-latex/unnamed-chunk-9-1.pdf}

Now we need to transform the list of `z' into an array with nsites x
nyears x spp

\begin{Shaded}
\begin{Highlighting}[]
\CommentTok{\# list to array}
\NormalTok{array\_sim\_data\_spp\_z }\OtherTok{\textless{}{-}} \FunctionTok{array}\NormalTok{(}\FunctionTok{as.numeric}\NormalTok{(}\FunctionTok{unlist}\NormalTok{(sim\_data\_spp\_z)), }
                              \AttributeTok{dim=}\FunctionTok{c}\NormalTok{(}\FunctionTok{length}\NormalTok{(gamma), }
\NormalTok{                                    nyears,}
\NormalTok{                                    nspp))}
\CommentTok{\# plot}
\FunctionTok{plot}\NormalTok{ (}\FunctionTok{colSums}\NormalTok{(array\_sim\_data\_spp\_z[,,}\DecValTok{1}\NormalTok{])}\SpecialCharTok{/}\FunctionTok{length}\NormalTok{(gamma), }\AttributeTok{type=}\StringTok{"b"}\NormalTok{, }
       \AttributeTok{ylab =} \StringTok{"Proportion of Occupied Sites"}\NormalTok{, }\AttributeTok{xlab =} \StringTok{"Time"}\NormalTok{,}
       \AttributeTok{ylim=}\FunctionTok{c}\NormalTok{(}\DecValTok{0}\NormalTok{,}\DecValTok{1}\NormalTok{),}
      \AttributeTok{col=}\StringTok{"gray"}\NormalTok{)}

\CommentTok{\# each spp}
\FunctionTok{lapply}\NormalTok{ (}\FunctionTok{seq}\NormalTok{ (}\DecValTok{1}\NormalTok{,nspp), }\ControlFlowTok{function}\NormalTok{ (i)}

  \FunctionTok{lines}\NormalTok{ (}\FunctionTok{seq}\NormalTok{(}\DecValTok{1}\NormalTok{,nyears),}
      \FunctionTok{colSums}\NormalTok{(array\_sim\_data\_spp\_z[,,i])}\SpecialCharTok{/}\FunctionTok{length}\NormalTok{(gamma), }\AttributeTok{type=}\StringTok{"b"}\NormalTok{, }
      \AttributeTok{col =} \FunctionTok{rgb}\NormalTok{ (}\DecValTok{0}\NormalTok{,}\DecValTok{1}\NormalTok{,}\DecValTok{1}\NormalTok{,}\AttributeTok{alpha=}\FloatTok{0.05}\NormalTok{) }
\NormalTok{      )}
\NormalTok{)}
\end{Highlighting}
\end{Shaded}

\includegraphics{README_files/figure-latex/unnamed-chunk-10-1.pdf} Then
draw the model, bundle data, and run the model

\begin{Shaded}
\begin{Highlighting}[]
\FunctionTok{cat}\NormalTok{(}\FunctionTok{readLines}\NormalTok{(}\StringTok{\textquotesingle{}dyn\_model\_vectorized\_no\_detection\_multispp.txt\textquotesingle{}}\NormalTok{), }\AttributeTok{sep =} \StringTok{\textquotesingle{}}\SpecialCharTok{\textbackslash{}n}\StringTok{\textquotesingle{}}\NormalTok{)}
\end{Highlighting}
\end{Shaded}

\begin{verbatim}

   
     model {
    
     #############################################################
    #                                                           #
    #                  Biological process                       #
    #                                                           #
    #############################################################
    
    
    # Site Occupancy Dynamics (Priors)
    for(i in 1:nsites) {
      gamma[i]~dunif(0,1)
      phi[i]~dunif(0,1)
    }
    
    psi1~dunif(0,1)     
         
    # Initial values
    for (i in 1:nsites) {
      for (s in 1:nspp) {
    
        p[i,1,s] <- psinit    # Initial occupancy probability
     
      }
    }
    

   ############      Model       #############
   for (i in 1:nsites) {
      for (s in 1:nspp) {
      
          y[i,1,s] ~ dbern(psi1) # Initial occupancy probability
    
              for (t in 2:nint){
            
                # model likelihood
                ### modeling dynamics conditional on previous time observed occurrence y
                
                muZ[i,t,s] <- y[i,t-1,s] *  phi[i] + ### if occupied, p of not getting extinct/persist in the next time
                          (1-y[i,t-1,s]) *  gamma[i] ###  if not occupied, p of originate in the next time
                
        # realized occurrence
            p[i,t,s] ~ dbern(muZ[i,t,s])      # Occupancy model
            y[i,t,s] ~ dbern(p[i,t,s])        # Observation model
    
        }#t
      }# spp
     }#i
    
    
    }## end of the model
    
    
    
    
\end{verbatim}

\begin{Shaded}
\begin{Highlighting}[]
\DocumentationTok{\#\# bundle data}
\FunctionTok{str}\NormalTok{(jags.data }\OtherTok{\textless{}{-}} \FunctionTok{list}\NormalTok{(}\AttributeTok{y =}\NormalTok{ array\_sim\_data\_spp\_z,}
                      \AttributeTok{nsites =} \FunctionTok{dim}\NormalTok{(array\_sim\_data\_spp\_z)[}\DecValTok{1}\NormalTok{],}
                      \AttributeTok{nint=} \FunctionTok{dim}\NormalTok{(array\_sim\_data\_spp\_z)[}\DecValTok{2}\NormalTok{],}
                      \AttributeTok{nspp =} \FunctionTok{dim}\NormalTok{(array\_sim\_data\_spp\_z)[}\DecValTok{3}\NormalTok{],}
                      \AttributeTok{psinit=}\NormalTok{psi1)}
\NormalTok{)}
\end{Highlighting}
\end{Shaded}

\begin{verbatim}
TRUE List of 5
TRUE  $ y     : num [1:12, 1:30, 1:100] 0 0 0 0 0 0 0 0 0 0 ...
TRUE  $ nsites: int 12
TRUE  $ nint  : int 30
TRUE  $ nspp  : int 100
TRUE  $ psinit: num 0.15
\end{verbatim}

\begin{Shaded}
\begin{Highlighting}[]
\DocumentationTok{\#\# Parameters to monitor}
\DocumentationTok{\#\# long form}
\NormalTok{params }\OtherTok{\textless{}{-}} \FunctionTok{c}\NormalTok{(}
  
  \StringTok{"psi1"}\NormalTok{,}
  \StringTok{"gamma"}\NormalTok{, }
  \StringTok{"phi"}\NormalTok{,}
  \StringTok{"p"}\NormalTok{,}
  \StringTok{"y"}
  
\NormalTok{)}

\DocumentationTok{\#\# MCMC settings}
\DocumentationTok{\#\#\#\#\#\#\#\#\#\#\#\#\#\#\#\#\#\#\#\#\#\#}
\NormalTok{na }\OtherTok{\textless{}{-}} \DecValTok{1000}\NormalTok{; nb }\OtherTok{\textless{}{-}} \DecValTok{5000}\NormalTok{; ni }\OtherTok{\textless{}{-}} \DecValTok{10000}\NormalTok{; nc }\OtherTok{\textless{}{-}} \DecValTok{3}\NormalTok{; nt }\OtherTok{\textless{}{-}} \DecValTok{10}

\CommentTok{\# MCMC runs}
\CommentTok{\# models}
\CommentTok{\# I ran and saved because it tooks a time}
\CommentTok{\#samples\_sims\_no\_det\_spp \textless{}{-} jags (data = jags.data, }
\CommentTok{\#                             parameters.to.save = params, }
\CommentTok{\#                             model.file = \#"dyn\_model\_vectorized\_no\_detection\_multispp.txt", }
\CommentTok{\#                             inits = NULL, }
\CommentTok{\#                             n.chains = nc, }
\CommentTok{\#                             n.thin = nt, }
\CommentTok{\#                             n.iter = ni, }
\CommentTok{\#                             n.burnin = nb, }
\CommentTok{\#                             DIC = T,  }
\CommentTok{\#                             parallel=F}
\CommentTok{\#)}

\CommentTok{\#save (samples\_sims\_no\_det\_spp, file = "samples\_sims\_no\_det\_spp.RData")}
\FunctionTok{load}\NormalTok{ (}\AttributeTok{file =} \StringTok{"samples\_sims\_no\_det\_spp.RData"}\NormalTok{)}
\end{Highlighting}
\end{Shaded}

Finally plot the results. As expected, running the model across many
taxa provide robustness to the estimates (note, compared to the previous
plots, the narrower credible intervals in the figures shown below) as
`gamma' and `phi' are updated while each taxon is analyzed. The model
adequately identified suboptimal, intermediate, and optimal sites for
taxon origination. All estimates of `gamma' and `phi' were within the
range of values used in the simulations.

\begin{Shaded}
\begin{Highlighting}[]
\CommentTok{\# plot the data}
\NormalTok{df\_res\_spp }\OtherTok{\textless{}{-}} \FunctionTok{do.call}\NormalTok{(rbind,gamma)}
\CommentTok{\# bind estimates}
\NormalTok{df\_res\_spp }\OtherTok{\textless{}{-}} \FunctionTok{cbind}\NormalTok{ (df\_res\_spp,}
                 \AttributeTok{sites =} \FunctionTok{seq}\NormalTok{(}\DecValTok{1}\NormalTok{,}\FunctionTok{nrow}\NormalTok{(df\_res\_spp)) ,}
                 \AttributeTok{gamma =}\NormalTok{ samples\_sims\_no\_det\_spp}\SpecialCharTok{$}\NormalTok{mean}\SpecialCharTok{$}\NormalTok{gamma,}
                 \AttributeTok{uci =} \FunctionTok{apply}\NormalTok{(samples\_sims\_no\_det\_spp}\SpecialCharTok{$}\NormalTok{sims.list}\SpecialCharTok{$}\NormalTok{gamma,}\DecValTok{2}\NormalTok{,quantile,}\FloatTok{0.975}\NormalTok{),}
                \AttributeTok{lci =} \FunctionTok{apply}\NormalTok{(samples\_sims\_no\_det\_spp}\SpecialCharTok{$}\NormalTok{sims.list}\SpecialCharTok{$}\NormalTok{gamma,}\DecValTok{2}\NormalTok{,quantile,}\FloatTok{0.025}\NormalTok{))}
\FunctionTok{colnames}\NormalTok{(df\_res\_spp)[}\DecValTok{1}\SpecialCharTok{:}\DecValTok{2}\NormalTok{]}\OtherTok{\textless{}{-}}\FunctionTok{c}\NormalTok{(}\StringTok{"inf\_range"}\NormalTok{,}\StringTok{"sup\_range"}\NormalTok{)}

\CommentTok{\# plot res simulations}

\FunctionTok{ggplot}\NormalTok{ (}\AttributeTok{data=}\FunctionTok{data.frame}\NormalTok{ (df\_res\_spp),}
        \FunctionTok{aes}\NormalTok{ (}\AttributeTok{x=}\NormalTok{sites,}\AttributeTok{y=}\NormalTok{gamma)) }\SpecialCharTok{+} 
  
  \FunctionTok{geom\_point}\NormalTok{(}\AttributeTok{col=}\StringTok{"red"}\NormalTok{, }\AttributeTok{size=}\DecValTok{2}\NormalTok{)}\SpecialCharTok{+}
  
  \FunctionTok{geom\_errorbar}\NormalTok{(}\FunctionTok{aes}\NormalTok{ (}\AttributeTok{ymin=}\NormalTok{lci,}\AttributeTok{ymax=}\NormalTok{ uci),}\AttributeTok{col=}\StringTok{"red"}\NormalTok{,}\AttributeTok{width=}\FloatTok{0.2}\NormalTok{,}\AttributeTok{linewidth=}\FloatTok{1.3}\NormalTok{)}\SpecialCharTok{+}

  \FunctionTok{geom\_errorbar}\NormalTok{(}\FunctionTok{aes}\NormalTok{ (}\AttributeTok{ymax=}\NormalTok{inf\_range,}\AttributeTok{ymin=}\NormalTok{ sup\_range),}\AttributeTok{width=}\FloatTok{0.2}\NormalTok{)}\SpecialCharTok{+}
  
  \FunctionTok{theme\_bw}\NormalTok{() }\SpecialCharTok{+} 
  
  \FunctionTok{ylab}\NormalTok{ (}\StringTok{"Origination probability"}\NormalTok{) }\SpecialCharTok{+} 
  
  \FunctionTok{xlab}\NormalTok{ (}\StringTok{"Sites"}\NormalTok{) }\SpecialCharTok{+} 
  
  \FunctionTok{scale\_x\_continuous}\NormalTok{(}\AttributeTok{breaks =} \FunctionTok{seq}\NormalTok{(}\DecValTok{1}\NormalTok{, }\DecValTok{12}\NormalTok{, }\AttributeTok{by =} \DecValTok{1}\NormalTok{))}
\end{Highlighting}
\end{Shaded}

\includegraphics{README_files/figure-latex/unnamed-chunk-13-1.pdf}

\begin{Shaded}
\begin{Highlighting}[]
\CommentTok{\# plot res simulations}
\CommentTok{\# bind estimates}

\NormalTok{df\_res\_phi\_spp }\OtherTok{\textless{}{-}} \FunctionTok{cbind}\NormalTok{ (}\AttributeTok{sites =} \FunctionTok{seq}\NormalTok{(}\DecValTok{1}\NormalTok{,}\FunctionTok{nrow}\NormalTok{(df\_res\_spp)) ,}
                 \AttributeTok{phi =}\NormalTok{ samples\_sims\_no\_det\_spp}\SpecialCharTok{$}\NormalTok{mean}\SpecialCharTok{$}\NormalTok{phi,}
                 \AttributeTok{uci =} \FunctionTok{apply}\NormalTok{(samples\_sims\_no\_det\_spp}\SpecialCharTok{$}\NormalTok{sims.list}\SpecialCharTok{$}\NormalTok{phi,}\DecValTok{2}\NormalTok{,quantile,}\FloatTok{0.975}\NormalTok{),}
                 \AttributeTok{lci =} \FunctionTok{apply}\NormalTok{(samples\_sims\_no\_det\_spp}\SpecialCharTok{$}\NormalTok{sims.list}\SpecialCharTok{$}\NormalTok{phi,}\DecValTok{2}\NormalTok{,quantile,}\FloatTok{0.025}\NormalTok{))}

\CommentTok{\# plot}
\FunctionTok{ggplot}\NormalTok{ (}\AttributeTok{data=}\FunctionTok{data.frame}\NormalTok{ (df\_res\_phi\_spp),}
        \FunctionTok{aes}\NormalTok{ (}\AttributeTok{x=}\NormalTok{sites,}\AttributeTok{y=}\NormalTok{phi)) }\SpecialCharTok{+} 
  
  \FunctionTok{geom\_point}\NormalTok{(}\AttributeTok{col=}\StringTok{"red"}\NormalTok{, }\AttributeTok{size=}\DecValTok{3}\NormalTok{)}\SpecialCharTok{+}
  
  \FunctionTok{geom\_errorbar}\NormalTok{(}\FunctionTok{aes}\NormalTok{ (}\AttributeTok{ymin=}\NormalTok{lci,}\AttributeTok{ymax=}\NormalTok{ uci),}\AttributeTok{col=}\StringTok{"red"}\NormalTok{,}\AttributeTok{width=}\FloatTok{0.2}\NormalTok{,}\AttributeTok{linewidth=}\FloatTok{1.3}\NormalTok{)}\SpecialCharTok{+}
  
  \CommentTok{\#geom\_errorbar(aes (ymin= min(phi), ymax=max(phi)),width=0.2)+}
  \FunctionTok{geom\_rect}\NormalTok{(}\FunctionTok{aes}\NormalTok{(}\AttributeTok{xmin =} \DecValTok{1}\NormalTok{, }\AttributeTok{xmax =} \DecValTok{12}\NormalTok{, }\AttributeTok{ymin =} \FloatTok{0.05}\NormalTok{, }\AttributeTok{ymax =} \FloatTok{0.1}\NormalTok{),}\AttributeTok{alpha=}\FloatTok{0.1}\NormalTok{) }\SpecialCharTok{+}

  \FunctionTok{theme\_bw}\NormalTok{() }\SpecialCharTok{+} 
  
  \FunctionTok{ylab}\NormalTok{ (}\StringTok{"Persistence probability"}\NormalTok{) }\SpecialCharTok{+} 
  
  \FunctionTok{xlab}\NormalTok{ (}\StringTok{"Sites"}\NormalTok{) }\SpecialCharTok{+} 
  
  \FunctionTok{scale\_x\_continuous}\NormalTok{(}\AttributeTok{breaks =} \FunctionTok{seq}\NormalTok{(}\DecValTok{1}\NormalTok{, }\DecValTok{12}\NormalTok{, }\AttributeTok{by =} \DecValTok{1}\NormalTok{))}
\end{Highlighting}
\end{Shaded}

\includegraphics{README_files/figure-latex/unnamed-chunk-14-1.pdf}

\end{document}
